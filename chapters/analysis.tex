After instrumenting the benchmarks to keep track of memory accesses, we found some interesting trends related to the statistics we chose to measure. While many of these trends were what we expected, others were more surprising and warranted further discussion.

\section{Alias Identification}
All of the alias analysis techniques identified above 90 percent of the pointers within the larger benchmarks, namely benchmarks that consisted of five or more source files. We attribute this to the larger number of virtual registers found within these benchmarks. The smaller benchmarks we used had more variable identification rates, ranging from about 60 to 75 percent. We expected that none of the benchmarks would have 100 percent identification due to pointer values within the source programs that did not belong to virtual registers, such as nested getelementptr instructions.

\section{Alias Misses}
Most of the benchmarks had relatively low alias miss rates, regardless of the alias analysis techniques. Over all of the instrumented benchmarks, low miss rates ranged from 0 to 7 percent. The number of alias misses was the same across all four analysis techniques for almost all of the benchmarks. Only three benchmarks showed differences in the number of alias misses - gzip, mcf, and crafty, and for these benchmarks, the differences were still small, typically consisting of 1 alias miss between different alias analyses. Otherwise, all of the alias analysis techniques tested were shown to be equally effective for most of the instrumented benchmarks.
