\chapter{Analysis}

\section{Results}
Table \ref{table:1} shows the number of aliases identified for each benchmark, separated by the type of alias analysis used.

\begin{table} [h!]
\centering
   \begin{tabular} {|c|c c c c|}
      \hline
      & Anders & Steens & ARC & Basic \\
      \hline
	   bzip2 & 624 & 624 & 624 & 624 \\
      \hline
	   gzip & 1170 & 1170 & 1170 & 1170 \\
      \hline
           mcf & 676 & 676 & 676 & 676 \\
      \hline
	   twolf & 9081 & 9081 & 9081 & 9081 \\
      \hline
	   parser & 3221 & 3221 & 3221 & 3221 \\
      \hline
	   vpr & 3701 & 3701 & 3701 & 3701 \\
      \hline
	   crafty & 4587 & 4587 & 4587 & 4587 \\
      \hline
	   sudoku & 77 & 77 & 77 & 77 \\
      \hline
	   matmul & 41 & 41 & 41 & 41 \\
      \hline
	   dict & 138 & 138 & 138 & 138 \\
      \hline
	   libc\_malloc & 171 & 171 & 171 & 171 \\
      \hline
	   libc\_malloc2 & 171 & 171 & 171 & 171 \\
      \hline
	   tcmalloc & 171 & 171 & 171 & 171 \\
      \hline
	   tree & 79 & 79 & 79 & 79 \\
      \hline
	   cycles & 27 & 27 & 27 & 27 \\
      \hline
   \end{tabular}
   \caption{Aliases Identified per Benchmark}
   \label{table:1}
\end{table}

\newpage

Table \ref{table:2} shows the number of alias misses for each benchmark, separated by the type of alias analysis used.

\begin{table} [h!]
\centering
   \begin{tabular} {|c|c c c c|}
      \hline
      & Anders & Steens & ARC & Basic \\
      \hline
	   bzip2 & 78 & 78 & 79 & 79 \\
      \hline
	   gzip & 41 & 41 & 41 & 41 \\
      \hline
           mcf & 103 & 104 & 104 & 104 \\
      \hline
	   twolf & 92 & 92 & 92 & 92 \\
      \hline
	   parser & 23 & 23 & 23 & 23 \\
      \hline
	   vpr & 218 & 218 & 218 & 218 \\
      \hline
	   crafty & 38 & 38 & 39 & 39 \\
      \hline
	   sudoku & 37 & 37 & 37 & 37 \\
      \hline
	   matmul & 13 & 13 & 13 & 13 \\
      \hline
	   dict & 106 & 106 & 106 & 106 \\
      \hline
	   libc\_malloc & 9 & 9 & 9 & 9 \\
      \hline
	   libc\_malloc2 & 9 & 9 & 9 & 9 \\
      \hline
	   tcmalloc & 0 & 0 & 0 & 0 \\
      \hline
	   tree & 42 & 42 & 42 & 42 \\
      \hline
	   cycles & 1 & 1 & 1 & 1 \\
      \hline
   \end{tabular}
   \caption{Alias Misses per Benchmark}
   \label{table:2}
\end{table}

\newpage

Table \ref{table:3} shows the alias miss rate for each benchmark, separated by the type of alias analysis used.

\begin{table} [h!]
\centering
   \begin{tabular} {|c|c c c c|}
      \hline
      & Anders & Steens & ARC & Basic \\
      \hline
	   bzip2 & 0.125 & 0.125 & 0.127 & 0.127 \\
      \hline
	   gzip & 0.035 & 0.035 & 0.035 & 0.035 \\
      \hline
           mcf & 0.152 & 0.154 & 0.154 & 0.154 \\
      \hline
	   twolf & 0.010 & 0,010 & 0,010 & 0,010 \\
      \hline
	   parser & 0.007 & 0.007 & 0.007 & 0.007 \\
      \hline
	   vpr & 0.059 & 0.059 & 0.059 & 0.059 \\
      \hline
	   crafty & 0.008 & 0.008 & 0.009 & 0.009 \\
      \hline
	   sudoku & 0.481 & 0.481 & 0.481 & 0.481 \\
      \hline
	   matmul & 0.317 & 0.317 & 0.317 & 0.317 \\
      \hline
	   dict & 0.768 & 0.768 & 0.768 & 0.768 \\
      \hline
	   libc\_malloc & 0.053 & 0.053 & 0.053 & 0.053 \\
      \hline
	   libc\_malloc2 & 0.053 & 0.053 & 0.053 & 0.053 \\
      \hline
	   tcmalloc & 0 & 0 & 0 & 0 \\
      \hline
	   tree & 0.532 & 0.532 & 0.532 & 0.532 \\
      \hline
	   cycles & 0.037 & 0.037 & 0.037 & 0.037 \\
      \hline
   \end{tabular}
   \caption{Alias Miss Rate per Benchmark}
   \label{table:3}
\end{table}

\newpage

Table \ref{table:4} shows the alias identification rate for each benchmark, separated by the type of alias analysis used.

\begin{table} [h!]
\centering
   \begin{tabular} {|c|c c c c|}
      \hline
      & Anders & Steens & ARC & Basic \\
      \hline
	   bzip2 & 0.990 & 0.990 & 0.990 & 0.990 \\
      \hline
	   gzip & 0.972 & 0.972 & 0.972 & 0.972 \\
      \hline
           mcf & 0.922 & 0.922 & 0.922 & 0.922 \\
      \hline
	   twolf & 0.998 & 0.998 & 0.998 & 0.998 \\
      \hline
	   parser & 0.993 & 0.993 & 0.993 & 0.993 \\
      \hline
	   vpr & 0.931 & 0.931 & 0.931 & 0.931 \\
      \hline
	   crafty & 0.998 & 0.998 & 0.998 & 0.998 \\
      \hline
	   sudoku & 0.606 & 0.606 & 0.606 & 0.606 \\
      \hline
	   matmul & 0.759 & 0.759 & 0.759 & 0.759 \\
      \hline
	   dict & 0.595 & 0.595 & 0.595 & 0.595 \\
      \hline
	   libc\_malloc & 0.966 & 0.966 & 0.966 & 0.966 \\
      \hline
	   libc\_malloc2 & 0.966 & 0.966 & 0.966 & 0.966 \\
      \hline
	   tcmalloc & 0.905 & 0.905 & 0.905 & 0.905 \\
      \hline
	   tree & 0.603 & 0.603 & 0.603 & 0.603 \\
      \hline
	   cycles & 0.871 & 0.871 & 0.871 & 0.871 \\
      \hline
   \end{tabular}
   \caption{Alias Identification Rate per Benchmark}
   \label{table:4}
\end{table}

\newpage

Table \ref{table:5} shows the mean and standard deviations of the alias lifetimes for each benchmark, and are independent of any alias analyses.

\begin{table} [h!]
\centering
   \begin{tabular} {|c|c c|}
      \hline
      & Mean & Std. Deviation \\
      \hline
	   bzip2 & 286487.505 & 843167.976 \\
      \hline
	   gzip & 252200.178 & 758121.815 \\
      \hline
           mcf & 175741.741 & 1002112.782 \\
      \hline
	   twolf & 68782.141 & 149007.719 \\
      \hline
	   parser & 3054.5 & 12379.139 \\
      \hline
	   vpr & 4762.618 & 24508.441 \\
      \hline
	   crafty & 276957.791 & 835831.453 \\
      \hline
	   sudoku & 184128.259 & 443716.239 \\
      \hline
	   matmul & 176101.094 & 748244.153 \\
      \hline
	   dict & 61732.615 & 172287.097 \\
      \hline
	   libc\_malloc & 31110.768.089 & 1330367.143 \\
      \hline
	   libc\_malloc2 & 279071.698 & 1263542.275 \\
      \hline
	   tcmalloc & 30.768.255 & 1407127.345 \\
      \hline
	   tree & 25.717 & 156.047 \\
      \hline
	   cycles & 1363.871 & 3351.797 \\
      \hline
   \end{tabular}
   \caption{Mean and Standard Deviation of Benchmark Lifetimes}
   \label{table:5}
\end{table}

\newpage

Table \ref{table:6} shows the mean and standard deviations of the allocation sizes for benchmarks that dynamically allocate memory, and are independent of any alias analyses.

\begin{table} [h!]
\centering
   \begin{tabular} {|c|c c c c c c|}
      \hline
      & matmul & dict & libc\_malloc & libc\_malloc2 & tree & cycles \\
      \hline
      Mean & 800 & 153745.578 & 1024 & 512 & 112.706 & 16 \\
      \hline
      Std. Dev. & 0 & 337535.376 & 0 & 0 & 111.295 & 0 \\
      \hline
   \end{tabular}
   \caption{Mean and Standard Deviation of Benchmark Allocation Sizes}
   \label{table:6}
\end{table}

Table \ref{table:7} shows the mean and standard deviations of the allocation lifetimes for benchmarks that dynamically allocate memory, and are independent of any alias analyses.

\begin{table} [h!]
\centering
   \begin{tabular} {|c|c c c c c c|}
      \hline
      & matmul & dict & libc\_malloc & libc\_malloc2 & tree & cycles \\
      \hline
      Mean & 8105 & 464768.632 & 0 & 0 & 12.353 & 45.352 \\
      \hline
      Std. Dev. & 94.130 & 258809.163 & 0 & 0 & 5.718 & 8.712 \\
      \hline
   \end{tabular}
   \caption{Mean and Standard Deviation of Benchmark Allocation Lifetimes}
   \label{table:7}
\end{table}

\section{Discussion of Results}
After instrumenting the benchmarks to keep track of memory accesses, we found some interesting trends related to the statistics we chose to measure. While many of these trends were what we expected, others were more surprising and warranted further discussion.

\subsection{Alias Identification}
As indicated by table \ref{table:4}, all of the alias analysis techniques identified above 90 percent of the pointers within the larger benchmarks, namely benchmarks that consisted of five or more source files. We attribute this to the larger number of virtual registers found within these benchmarks; larger programs have a higher number of memory access instructions with single virtual register operands that can be examined by alias analyses, and despite the increased program complexity, there is a lower proportion of irregular operands that are not identified. The smaller benchmarks we used had more variable identification rates, ranging from about 60 to 75 percent. We expected that none of the benchmarks would have 100 percent identification due to pointer values within the source programs that did not belong to virtual registers, such as nested getelementptr instructions. This is because the alias analyses techniques examined primarily focus on virtual registers, instead of irregular operands.

\subsection{Alias Misses}
Most of the benchmarks had relatively low alias miss rates, regardless of the alias analysis techniques. Over all of the instrumented benchmarks, low miss rates ranged from 0 to 7 percent. The number of alias misses was the same across all four analysis techniques for almost all of the benchmarks. Only three benchmarks showed differences in the number of alias misses - bzip2, mcf, and crafty, and for these benchmarks, the differences were still small, typically consisting of one alias miss between different alias analyses. Otherwise, all four alias analysis techniques tested were shown to be equally effective for most of the instrumented benchmarks.

\subsubsection{bzip2}

\subsubsection{mcf}

\subsubsection{crafty}

\subsection{Alias Miss Rates}
Some benchmarks had high alias miss rates. In this case, we considered high miss rates as over 20 percent on average across all alias analyses. These benchmarks include the sudoku, matmul, dict, and tree benchmarks. After examining the original source files for these benchmarks, we provide possible reasons for the unusually high miss rates.

\subsubsection{Sudoku}
The sudoku benchmark calculates the solution for nine example sudoku problems that are considered difficult for machines to solve by brute force. Across all alias analyses, this benchmark has an average miss rate of about 48 percent. The solving algorithm uses a binary matrix, declared as a two-dimensional array, representing a series of constraints that make up a valid sudoku puzzle, and iteratively tries different sets of constraints until a valid solution is found. As expected, the brute-force algorithm often requires significant backtracking, iteration, and conditional statements related to updating the binary matrix, resulting in numerous basic blocks within the associated LLVM files. Because none of the tested alias analysis techniques are flow-sensitive, they would not be expected to easily identify many of the resulting pointers for possible aliasing within the solution function.

\subsection{Matrix Multiplication}
The matmul benchmark multiplies two randomly-generated 100 by 100 matrices together, and has an average miss rate of about 32 percent. The matrices are dynamically allocated, providing a large number of potential pointers into the matrices to alias with. The classic multiplication algorithm utilizes three nested loops to calculate the resulting matrix. As with the sudoku benchmark, the prominent amount of iteration within this program introduced additional program flow that none of the given alias analyses could reliably address. Compared with the sudoku benchmark, the larger amount of memory used within this benchmark did not result in a higher percentage of alias misses; even though matrix multiplication still requires a large amount of memory for the input matrices, the loop body itself features fewer flow-sensitive statements, resulting in a lower overall miss rate.

\subsubsection{Dictionary}
The dict benchmark creates a hash table to store a series of input strings, and the resulting alias miss rate is about 77 percent. The significantly higher miss rate can be attributed to the logic related to updating and maintaining the hash table's entries. As with the Sudoku solving algorithm, the logic related to inserting variable-sized entries into the hash table rely upon several loops and conditional statements, primarily for traversing the hash table's linked-list buckets and updating various attributes for the hash table. Unlike the Sudoku benchmark, accesses to the hash table are also more variable, and updates to the hash table that resize the hash table may occur, potentially invalidating previous aliases. Thus, hash tables are particularly difficult for the selected alias analysis techniques to effectively measure.

\subsubsection{Tree}
The tree benchmark creates a search tree from a series of input strings that has branches based on common prefixes of one or more letters, and traverses this tree to retrieve specific strings requested by the user. This benchmark has a miss rate of about 53 percent, which is likely related to the iteration required to traverse the tree, along with the dynamic allocation of variable-sized amounts of dynamic memory for each tree's corresponding letters. Additionally, combining tree nodes based on common prefixes also affects possible pointer aliases, in a way that is similar to the previous dict benchmark.

\subsection{Alias Lifetimes}

\subsection{Allocation Sizes}

\subsection{Allocation Lifetimes}
