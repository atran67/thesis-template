\chapter{Analysis}

After instrumenting the benchmarks to keep track of memory accesses, we found some interesting trends related to the statistics we chose to measure. While many of these trends were what we expected, others were more surprising and warranted further discussion.

\section{Pointer Identification}
Table \ref{table:1} shows the number of pointer operands reported by alias analyses for each benchmark, the total number of pointers, and the identification rate found for each benchmark. For all the benchmarks, all four alias analyses identified the same number of aliases.

\begin{table} [h!]
\centering
   \begin{tabular} {|c|c c c|}
      \hline
	   & Identified Pointers & Total Pointers & Identification Rate \\
      \hline
	   bzip2 & 624 & 630 & 0.990 \\
      \hline
	   gzip & 1170 & 1204 & 0.972 \\
      \hline
	   mcf & 676 & 733 & 0.922 \\
      \hline
	   twolf & 9081 & 9099 & 0.998 \\
      \hline
	   parser & 3221 & 3243 & 0.993 \\
      \hline
	   vpr & 3701 & 3975 & 0.931 \\
      \hline
	   crafty & 4587 & 4598 & 0.993 \\
      \hline
	   sudoku & 77 & 127 & 0.606 \\
      \hline
	   matmul & 41 & 54 & 0.759 \\
      \hline
	   dict & 138 & 232 & 0.594 \\
      \hline
	   libc\_malloc & 171 & 177 & 0.966 \\
      \hline
	   libc\_malloc2 & 171 & 177 & 0.966 \\
      \hline
	   tcmalloc & 171 & 177 & 0.966 \\
      \hline
	   tree & 79 & 131 & 0.603 \\
      \hline
	   cycles & 27 & 31 & 0.871 \\
      \hline
   \end{tabular}
   \caption{Pointers Identified, Total Pointers, and Identification Rate per Benchmark}
   \label{table:1}
\end{table}

As indicated by Table \ref{table:1}, all of the alias analysis techniques identified above 90 percent of the pointer operands within the larger benchmarks, namely benchmarks that consisted of five or more source files. We attribute this to the larger number of virtual registers found within these benchmarks; larger programs have a higher number of memory access instructions with single virtual register operands that can be examined by alias analyses, and despite the increased program complexity, there is a lower proportion of irregular operands that are not identified. The smaller benchmarks we used had more variable identification rates, ranging from about 60 to 75 percent. We expected that none of the benchmarks would have 100 percent pointer identification, where every memory address is mapped to a pointer operand that is accounted for in an alias analysis, due to pointer values within the source programs that did not belong to virtual registers, such as nested getelementptr instructions. This is because the alias analysis techniques examined primarily focus on virtual registers, instead of irregular operands, so not all memory accesses would be covered by these analyses.

\section{Alias Misses}
Table \ref{table:2} shows the number of alias misses for each benchmark, separated by the type of alias analysis used. When calculating alias misses, we only consider the first memory address assigned to that pointer operand. Thus, if two pointers are said to alias, and one of these pointers has multiple differing memory addresses over the span of the program, this is considered only one alias miss. Two pointers declared as possible aliases are also considered a miss if they have different memory addresses.

\begin{table} [h!]
\centering
   \begin{tabular} {|c|c c c c|}
      \hline
      & Anders & Steens & ARC & Basic \\
      \hline
	   bzip2 & 78 & 78 & 79 & 79 \\
      \hline
	   gzip & 41 & 41 & 41 & 41 \\
      \hline
           mcf & 103 & 104 & 104 & 104 \\
      \hline
	   twolf & 92 & 92 & 92 & 92 \\
      \hline
	   parser & 23 & 23 & 23 & 23 \\
      \hline
	   vpr & 218 & 218 & 218 & 218 \\
      \hline
	   crafty & 38 & 38 & 39 & 39 \\
      \hline
	   sudoku & 37 & 37 & 37 & 37 \\
      \hline
	   matmul & 13 & 13 & 13 & 13 \\
      \hline
	   dict & 106 & 106 & 106 & 106 \\
      \hline
	   libc\_malloc & 9 & 9 & 9 & 9 \\
      \hline
	   libc\_malloc2 & 9 & 9 & 9 & 9 \\
      \hline
	   tcmalloc & 0 & 0 & 0 & 0 \\
      \hline
	   tree & 42 & 42 & 42 & 42 \\
      \hline
	   cycles & 1 & 1 & 1 & 1 \\
      \hline
   \end{tabular}
   \caption{Alias Misses per Benchmark}
   \label{table:2}
\end{table}

Most of the benchmarks had relatively low alias miss rates, regardless of the alias analysis techniques. Over all of the instrumented benchmarks, low miss rates ranged from 0 to 7 percent. The number of alias misses was the same across all four analysis techniques for almost all of the benchmarks. Only three benchmarks showed differences in the number of alias misses - bzip2, mcf, and crafty, and for these benchmarks, the differences were still small, typically consisting of one alias miss between different alias analyses. Otherwise, all four alias analysis techniques tested were shown to be equally effective for most of the instrumented benchmarks.

It's important to remember that the alias analyses are run on functions in LLVM source files, which consist of a one-dimensional array of blocks of statements. High-level control flow statements are translated to block separations, and SSA ensures that updates to existing values produce new virtual registers that may be examined in an alias analysis. Although aliases may still exist across blocks, the amount of ambiguity caused by program flow and stateful updates is reduced at this level of the program. This may explain why different analyses are often equally effective at this level, when they would have more pronounced differences in a more abstract programming language.

\section{Alias Miss Rates}
Table \ref{table:3} shows the alias miss rate for each benchmark, separated by the type of alias analysis used.

\begin{table} [h!]
\centering
   \begin{tabular} {|c|c c c c|}
      \hline
      & Anders & Steens & ARC & Basic \\
      \hline
	   bzip2 & 0.125 & 0.125 & 0.127 & 0.127 \\
      \hline
	   gzip & 0.035 & 0.035 & 0.035 & 0.035 \\
      \hline
           mcf & 0.152 & 0.154 & 0.154 & 0.154 \\
      \hline
	   twolf & 0.010 & 0.010 & 0.010 & 0.010 \\
      \hline
	   parser & 0.007 & 0.007 & 0.007 & 0.007 \\
      \hline
	   vpr & 0.059 & 0.059 & 0.059 & 0.059 \\
      \hline
	   crafty & 0.008 & 0.008 & 0.009 & 0.009 \\
      \hline
	   sudoku & 0.481 & 0.481 & 0.481 & 0.481 \\
      \hline
	   matmul & 0.317 & 0.317 & 0.317 & 0.317 \\
      \hline
	   dict & 0.768 & 0.768 & 0.768 & 0.768 \\
      \hline
	   libc\_malloc & 0.053 & 0.053 & 0.053 & 0.053 \\
      \hline
	   libc\_malloc2 & 0.053 & 0.053 & 0.053 & 0.053 \\
      \hline
	   tcmalloc & 0 & 0 & 0 & 0 \\
      \hline
	   tree & 0.532 & 0.532 & 0.532 & 0.532 \\
      \hline
	   cycles & 0.037 & 0.037 & 0.037 & 0.037 \\
      \hline
   \end{tabular}
   \caption{Alias Miss Rate per Benchmark}
   \label{table:3}
\end{table}

Some benchmarks had high alias miss rates. In this case, we considered high miss rates as over 20 percent on average across all alias analyses. These benchmarks include the sudoku, matmul, dict, and tree benchmarks. After examining the original source files for these benchmarks, we provide possible reasons for the unusually high miss rates.

\subsection{Sudoku}
The sudoku benchmark calculates the solution for nine example sudoku problems that are considered difficult for machines to solve by brute force. Across all alias analyses, this benchmark has an average miss rate of about 48 percent. The solving algorithm uses a binary matrix, declared as a two-dimensional array, representing a series of constraints that make up a valid sudoku puzzle, and iteratively tries different sets of constraints until a valid solution is found. As expected, the brute-force algorithm often requires significant backtracking, iteration, and conditional statements related to updating the binary matrix, resulting in numerous basic blocks within the associated LLVM files. Because none of the tested alias analysis techniques are flow-sensitive, they would not be expected to easily identify many of the resulting pointers for possible aliasing within the solution function.

For this benchmark, the function \texttt{sd\_update} has the highest potential for alias misses; as the name suggests, this function updates the binary matrix used to represent the sudoku puzzle, and is frequently called throughout the benchmark. The C implementation of this function is shown in Figure \ref{fig:sdc}.

\begin{figure} [h!]
  \lstinputlisting[basicstyle=\ttfamily\tiny, language=C, numbers=left] {figures/sd_update.c}
  \caption{\texttt{sd\_update} implementation from the Sudoku Benchmark}
  \label{fig:sdc}
\end{figure}

The statement at the beginning line 6 of Figure \ref{fig:sdc} updates the state vectors and the binary matrix by iterating through the rows and columns. Referencing the matrix itself requires indirection into a structure and a field reference, followed by multiple indexes into an array. Consequently, this is translated into multiple getelementptr and load instructions to access the appropriate entry in the matrix. Accessing the array in this manner is fairly common, and occurs several times at varying levels of nested loops. The statement translated into the LLVM IR is shown in Figure \ref{fig:sdll}.

\begin{figure} [h!]
  \lstinputlisting[basicstyle=\ttfamily\tiny, firstline=758, lastline=776, numbers=left] {figures/sudoku_v1.ll}
  \caption{LLVM instructions from the Sudoku Benchmark}
  \label{fig:sdll}
\end{figure}

All of the memory accesses in Figure \ref{fig:sdll} are contained within a loop, and cannot be analyzed well by flow-insensitive analyses. Specifically, the Steensgaard analysis cannot determine the aliases well because each statement is only processed once, which does not properly reflect the iteratve nature of this block; any aliases for the matrix entry to values outside of the loop or inside other loops would miss. As an example, for pointer register \texttt{\%37}, the Steensgaard Analysis gives registers \texttt{\%57}, \texttt{\%79}, \texttt{\%97}, \texttt{\%103}, and \texttt{\%158} as possible aliases, but all of these pointers belong to blocks within different nested loops based on their respective branches, resulting in higher miss rates. Similarly, the Andersen analysis is able to gather constraints for statements within loops, but cannot reflect changes before and after due to iteration within the program. Ultimately, because this iterative access pattern occurs so frequently in this function, it contributes to higher miss rates.

\subsection{Matrix Multiplication}
The matmul benchmark multiplies two randomly-generated 100 by 100 matrices together, and has an average miss rate of about 32 percent. The matrices are dynamically allocated, providing a large number of potential pointers into the matrices to alias with. The classic multiplication algorithm utilizes three nested loops to calculate the resulting matrix. As with the sudoku benchmark, the prominent amount of iteration within this program introduced additional program flow that none of the given alias analyses could reliably address due to treating array elements as aliases with the base array. Compared with the sudoku benchmark, the larger amount of memory used within this benchmark did not result in a higher percentage of alias misses; even though matrix multiplication still requires a large amount of memory for the input matrices, the loop body itself features fewer flow-sensitive statements, resulting in a lower overall miss rate. The C implementation of the matrix multiplication function is shown below.

\begin{figure} [h!]
  \lstinputlisting[language=C, basicstyle=\ttfamily\tiny, numbers=left] {figures/matmul.c}
  \caption{Matrix Multiplication Function}
  \label{fig:mmc}
\end{figure}

Within the matrix multiplication function in Figure \ref{fig:mmc}, there are two areas of interest: the transposing of the second matrix on lines 6 to 9, and the calculation of each entry on lines 13 to 16. The body of the inner loop on line 9, translated to LLVM IR, is shown below.

\begin{figure} [h!]
  \lstinputlisting[basicstyle=\ttfamily\tiny, firstline=233, lastline=250, numbers=left] {figures/matmul_v1.ll}
  \caption{Transpose Loop translated to LLVM}
  \label{fig:mmll}
\end{figure}

Each access into the matrix requires two getelementptr instructions, as on lines 4 and 13, and two load instructions, as on lines 5 and 14, to index twice into the matrix. Additionally, the matrix pointer and the indices used are loaded each time, and are updated after every loop iteration. Combined with the single store instruction for updating the matrix, this block alone has twelve memory access instructions out of the nineteen total instructions, none of which can be analyzed well by flow-insensitive alias analyses because they are within a doubly-nested for loop. As with the sudoku benchmark, the typing rules for both the Steensgaard and Andersen analyses only process the statements in the loop body once, and thus, cannot properly reflect the changes caused by loop iteration. For this function, the Steensgaard Analysis lists registers \texttt{\%68}, \texttt{\%78}, and \texttt{\%88} as possible aliases of pointer register \texttt{\%37}, where all of these pointers exist in different loops. The loop body from lines 13 to 16 of Figure \ref{fig:mmc} suffers from similar problems. The translated loop is shown in Figure \ref{fig:mmll2}, and has a similar concentration of memory access instructions as the previous loop body. Note that register \texttt{\%78} exists in this loop, leading to mismatches with earlier aliases.

\begin{figure} [h!]
  \lstinputlisting[basicstyle=\ttfamily\tiny, firstline=312, lastline=346, numbers=left] {figures/matmul_v1.ll}
  \caption{Inner Loop translated to LLVM}
  \label{fig:mmll2}
\end{figure}

\subsection{Dictionary}
The dict benchmark creates a hash table to store a series of input strings, and the resulting alias miss rate is about 77 percent. The significantly higher miss rate can be attributed to the logic related to updating and maintaining the hash table's entries. As with the Sudoku solving algorithm, the logic related to inserting variable-sized entries into the hash table is implemented as several loops and conditional statements, primarily for traversing the hash table's array buckets and updating various attributes for the hash table. Unlike the Sudoku benchmark, accesses to the hash table are also more variable, and updates to the hash table that resize the hash table may occur, potentially invalidating previous aliases. Thus, hash tables are particularly difficult for the selected alias analysis techniques to effectively characterize. Within the main program loop, each occurrence of a string is inserted into the hash table, as shown in Figure \ref{fig:dictc}, using the macros \texttt{kh\_put}, \texttt{kh\_val}, and \texttt{kh\_key} to access the hash table.

\begin{figure} [h!]
  \lstinputlisting[basicstyle=\ttfamily\tiny, language=C, firstline=21, lastline=36, numbers=left] {figures/dict_v1.c}
  \caption{Dictionary Insertion in C}
  \label{fig:dictc}
\end{figure}

When lines 10 to 12 in Figure \ref{fig:dictc} are translated to LLVM, \texttt{kh\_key} and \texttt{kh\_val} are treated as array accesses. The arrays of keys and values are fields in the hash table struct, so their offsets are fixed when accessing them from the struct via getelementptr and load instructions. However, the array indices are based on the hash value of the string, which results in aliasing issues caused by unpredictable accesses. The LLVM IR for lines 10 to 12 is shown in Figure \ref{fig:dictll}.

\begin{figure} [h!]
  \lstinputlisting[basicstyle=\ttfamily\tiny, firstline=140, lastline=157, numbers=left] {figures/dict_v1.ll}
  \caption{Hashing calls translated to LLVM}
  \label{fig:dictll}
\end{figure}

The hash value is calculated by the function \texttt{kh\_put\_str}, which replaces the macro \texttt{kh\_put}. This function utilizes multiple conditional statements and memory accesses into the input string to produce the hash value. Unlike the earlier benchmarks, most of the memory accesses refer to the same struct using getelementptr, but with different offsets to produce distinct byte pointers. The inability to distinguish between these varying offsets makes this function difficult to analyze effectively using flow-insensitive alias analyses. One of the blocks illustrating these problems is shown in Figure \ref{fig:dictll2}. Both the Steensgaard and the Andersen analyses treats accesses to pointer register \texttt{\%4} with different offsets, such as \texttt{\%110} with an offset of 5, and \texttt{\%280} with an offset of 4, across multiple blocks, as aliases based on the original struct pointer, leading to a high number of misses.

\begin{figure} [h!]
  \lstinputlisting[basicstyle=\ttfamily\tiny, firstline=247, lastline=247] {figures/dict_v1.ll}
  \lstinputlisting[basicstyle=\ttfamily\tiny, firstline=384, lastline=386] {figures/dict_v1.ll}
  \lstinputlisting[basicstyle=\ttfamily\tiny, firstline=608, lastline=610] {figures/dict_v1.ll}
  \caption{Excerpts from \texttt{kh\_put\_str}}
  \label{fig:dictll2}
\end{figure}

\subsection{Tree}
The tree benchmark creates a search tree from a series of input strings that has branches based on common prefixes of one or more letters, and traverses this tree to retrieve specific strings requested by the user. This benchmark has a miss rate of about 53 percent, which is related to the iteration required to traverse the tree, along with the dynamic allocation of variable-sized amounts of dynamic memory for each tree's corresponding letters. The Andersen and Steensgaard analyses both identify the tree node pointers in registers \texttt{\%14}, \texttt{\%26}, \texttt{\%66} and \texttt{\%137} as aliases, but the latter three are pointers that are updated within separate loops. Additionally, byte pointers based off of offsets from the struct pointers, such as registers \texttt{\%83}, \texttt{\%117}, and \texttt{\%156}, are treated incorrectly as aliases, in a way that is similar to the previous dict benchmark.

References to recursive data structures, like trees, are first loaded from the appropriate struct field offsets, and are treated as a separate pointer for subsequent loads and stores. When these fields are repeatedly updated, as in loops, this causes aliasing issues. Excerpts from the function for collapsing tree nodes, which involves such tree pointer traversal and reassignment, is shown below in both C, and the LLVM IR, illustrating these potential issues.

\begin{figure} [h!]
  \lstinputlisting[basicstyle=\ttfamily\tiny, language=C, firstline=151, lastline=173, numbers=left] {figures/tree.c}
  \caption{Excerpt from the tree benchmark in C}
  \label{fig:treec}
\end{figure}

\begin{figure} [h!]
  \lstinputlisting[basicstyle=\ttfamily\tiny, firstline=649, lastline=667, numbers=left] {figures/tree.ll}
  \caption{Excerpt from the tree benchmark in LLVM}
  \label{fig:treell}
\end{figure}

\section{Pointer Lifetimes}
Table \ref{table:4} shows the mean and standard deviations of the pointer lifetimes for each benchmark, and are independent of any alias analyses.

\begin{table} [h!]
\centering
   \begin{tabular} {|c|c c|}
      \hline
      & Mean & Std. Deviation \\
      \hline
	   bzip2 & 286487.505 & 843167.976 \\
      \hline
	   gzip & 252200.178 & 758121.815 \\
      \hline
           mcf & 175741.741 & 1002112.782 \\
      \hline
	   twolf & 68782.141 & 149007.719 \\
      \hline
	   parser & 3054.5 & 12379.139 \\
      \hline
	   vpr & 4762.618 & 24508.441 \\
      \hline
	   crafty & 276957.791 & 835831.453 \\
      \hline
	   sudoku & 184128.259 & 443716.239 \\
      \hline
	   matmul & 176101.094 & 748244.153 \\
      \hline
	   dict & 61732.615 & 172287.097 \\
      \hline
	   libc\_malloc & 311113.089 & 1330367.143 \\
      \hline
	   libc\_malloc2 & 279071.698 & 1263542.275 \\
      \hline
	   tcmalloc & 313728.255 & 1407127.345 \\
      \hline
	   tree & 25.717 & 156.047 \\
      \hline
	   cycles & 1363.871 & 3351.797 \\
      \hline
   \end{tabular}
   \caption{Mean and Standard Deviation of Benchmark Pointer Lifetimes (# of Instrumented Instructions)}
   \label{table:4}
\end{table}

The average pointer lifetimes were significantly higher for the larger benchmarks. We suspect that this is due to variables remaining throughout the span of the program, such as input data to be processed, or structs maintaining program state. At the same time, the standard deviation of the pointer lifetimes was also high, reflecting varying access patterns. This suggests that a large number of pointers are being used within a short timespan, likely within loops. The standard deviations are much higher than their respective averages, implying that a large number of these short range pointers exist, while the averages are skewed toward longer-lived pointers.

