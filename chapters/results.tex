\chapter{Results}

Table \ref{table:1} shows the number of aliases identified for each benchmark, separated by the type of alias analysis used.

\begin{table}
\centering
   \begin{tabular} {|c|c c c c c c c c c c c c c c c|}
      \hline
      & bzip2 & gzip & mcf & twolf & parser & vpr & crafty & sudoku & matmul & dict & libc_malloc & libc_malloc2 & tcmalloc & tree & cycles \\
      \hline
      Anders & 624 & 1170 & 676 & 9081 & 3221 & 3701 & 4587 & 77 & 41 & 138 & 171 & 171 & 171 & 79 & 27 \\
      \hline
      Steens & 624 & 1170 & 676 & 9081 & 3221 & 3701 & 4587 & 77 & 41 & 138 & 171 & 171 & 171 & 79 & 27 \\
      \hline
      ARC & 624 & 1170 & 676 & 9081 & 3221 & 3701 & 4587 & 77 & 41 & 138 & 171 & 171 & 171 & 79 & 27 \\
      \hline
      Basic & 624 & 1170 & 676 & 9081 & 3221 & 3701 & 4587 & 77 & 41 & 138 & 171 & 171 & 171 & 79 & 27 \\
      \hline
   \end{tabular}
   \caption{Aliases Identified per Benchmark}
   \label{table:1}
\end{table}

Table \ref{table:2} shows the number of alias misses for each benchmark, separated by the type of alias analysis used.

\begin{table}
\centering
   \begin{tabular} {|c|c c c c c c c c c c c c c c c|}
      \hline
      & bzip2 & gzip & mcf & twolf & parser & vpr & crafty & sudoku & matmul & dict & libc_malloc & libc_malloc2 & tcmalloc & tree & cycles \\
      \hline
      Anders
      \hline
      Steens
      \hline
      ARC
      \hline
      Basic
      \hline
   \end{tabular}
   \caption{Alias Misses per Benchmark}
   \label{table:2}
\end{table}

Table \ref{table:3} shows the alias miss rate for each benchmark, separated by the type of alias analysis used.

\begin{table}
\centering
   \begin{tabular} {|c|c c c c c c c c c c c c c c c|}
      \hline
      & bzip2 & gzip & mcf & twolf & parser & vpr & crafty & sudoku & matmul & dict & libc_malloc & libc_malloc2 & tcmalloc & tree & cycles \\
      \hline
      Anders
      \hline
      Steens
      \hline
      ARC
      \hline
      Basic
      \hline
   \end{tabular}
   \caption{Alias Miss Rate per Benchmark}
   \label{table:3}
\end{table}

Table \ref{table:4} shows the mean and standard deviations of the alias lifetimes for each benchmark, and are independent of any alias analyses.

\begin{table}
\centering
   \begin{tabular} {|c|c c c c c c c c c c c c c c c|}
      \hline
      & bzip2 & gzip & mcf & twolf & parser & vpr & crafty & sudoku & matmul & dict & libc_malloc & libc_malloc2 & tcmalloc & tree & cycles \\
      \hline
      Mean
      \hline
      Std. Dev.
      \hline
   \end{tabular}
   \caption{Mean and Standard Deviation of Benchmark Lifetimes}
   \label{table:4}
\end{table}

Table \ref{table:5} shows the mean and standard deviations of the allocation sizes for benchmarks that dynamically allocate memory, and are independent of any alias analyses.

\begin{table}
\centering
   \begin{tabular} {|c|c c c c c c|}
      \hline
      & matmul & dict & libc_malloc & libc_malloc2 & tree & cycles \\
      \hline
      Mean
      \hline
      Std. Dev.
      \hline
   \end{tabular}
   \caption{Mean and Standard Deviation of Benchmark Allocation Sizes}
   \label{table:5}
\end{table}

Table \ref{table:5} shows the mean and standard deviations of the allocation lifetimes for benchmarks that dynamically allocate memory, and are independent of any alias analyses.

\begin{table}
\centering
   \begin{tabular} {|c|c c c c c c|}
      \hline
      & matmul & dict & libc_malloc & libc_malloc2 & tree & cycles \\
      \hline
      Mean
      \hline
      Std. Dev.
      \hline
   \end{tabular}
   \caption{Mean and Standard Deviation of Benchmark Allocation Lifetimes}
   \label{table:6}
\end{table}
