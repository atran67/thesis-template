\chapter{Validation}

As an empirical study, we gather statistics from the instrumented programs to gather information about their memory access patterns. We organize the runtime data we retrieve and combine it with our alias analyses to produce meaningful metrics for each of the benchmarks.

\section{Alias Misidentification}
Because alias analyses are static, their accuracy cannot be confirmed until the program is run. By mapping the outputted memory addresses back to the original operands in their respective LLVM files, we can confirm whether or not two pointers alias. We define an alias miss as two operands that are stated by an alias analyses to alias, but have differing memory addresses at runtime. We also consider two aliases to miss if one of those aliases has an unknown address at runtime. Thus, the alias miss rate is defined as the number of alias misses over the total number of aliases found.

\section{Alias Lifetime}
To better examine memory access patterns, we want to measure how long any given operands in a program refer to the same area of memory. We define alias lifetime as how long, in terms of instrumented instructions executed, an operand, such as a virtual register, refers to the same memory address before being changed. We expect operands within loops or conditional statements to have shorter lifetimes due to rapid updates and reassignment, while variables used throughout the span of functions are expected to have longer lifetimes.

\section{Allocation Size and Lifetime}
We gather the specified sizes, in terms of bytes, when dynamically allocating memory to get a better sense of what is being allocated over time. More consistent allocation sizes may imply repeated use of data structures, such as linked lists, or types, such as common uses of arrays; conversely, variable allocation sizes may reflect more dynamic uses of memory, such as storing strings. Similarly, we measure allocation lifetime in terms of "ticks" specified by the standard library clock function call.
