As an empirical study, we gather statistics from the instrumented programs to gather information about their memory access patterns. We organize the runtime data we retrieve and combine it with our alias analyses to produce meaningful metrics for each of the benchmarks.

\section{Alias Misses}
Because alias analyses are static, their accuracy cannot be confirmed until the program is run. By mapping the outputted memory addresses back to the original operands in their respective LLVM files, we can confirm whether or not two pointers alias. We define an alias miss as an two operands that are stated by an alias analyses to alias, but have differing memory addresses at runtime. We also consider two aliases to miss if one of those aliases has an unknown address at runtime. Thus, the alias miss rate is defined as the number of alias misses over the total number of aliases found Thus, the alias miss rate for an alias analysis is defined as the number of alias misses over the total number of aliases found by that analysis.
