\chapter{Related Work}

Much work has been done in the area of alias analysis, both in proposing new alias analysis techniques, and for evaluating such techniques. Because alias analysis is an undecidable problem \cite{undecidable}, many potential avenues exist for developing more precise or efficient approximations.

\section{Current Alias Analyses}
Several existing alias analyses are currently used as part of contemporary compiler optimizations. These analyses take different approaches in identifying aliases, and thus vary in terms of efficiency and effectiveness. We are interested in the alias analysis implementations that target LLVM instructions. The following example program is used to demonstrate some of the following alias analyses.

\begin{figure} [h]
   \lstinputlisting[language=C, basicstyle=\ttfamily\small, numbers=left] {figures/codex.c}
   \caption{Example Program for Evaluating Alias Analyses}
   \label{fig:codex}
\end{figure}

\subsection{Andersen Analysis}
Andersen's alias analysis is an analysis technique for determining pointer aliases within functions without considering program flow \cite{Andersen}. Andersen analysis provides set notation and type inference rules meant for the C programming language. Pointers are initially stated to be part of specific types of pointers, such as global variables, dynamically allocated memory, and function parameters. Additional type inference rules are used to represent different operations performed with pointers, such as dereferencing, assignment, and type casting. These rules are used to generate set constraints for the pointer values within a function. For alias analysis across function calls, Andersen analysis uses static call graphs and additional inference rules to generate context-sensitive constraints. The generated constraints for the aliases are solved by a using set of rewriting rules for type normalization and propagation; the constraint solving algorithm's time complexity is polynomial in terms of program size.

In the example program in Figure \ref{fig:codex}, shown above, constraints are generated for the statements on lines 5, 6, 7, 12, 15, and 18, which all involve assignment of pointer values. After resolving these constraints, \texttt{x} is found to reference \texttt{a}, \texttt{y} is found to reference \texttt{b} and an unknown pointer, and \texttt{z} is found to reference \texttt{a} and \texttt{b}. The conditional statement only collects constraints within its then and else clauses, and as statements are processed, possible aliases for pointers are only added; this can result in some inaccuracies, such as with analyzing \texttt{y} and \texttt{z}. Additionally, expressions with binary operators, such as those in lines 12 and 15, are dependent on functions that generate constraints based on the operator and the operands; for the first statement, the constraint refers to an unknown pointer due to the operator subtracting an integer from a pointer to a single variable, leading into an unknown area, and for the second statement, the constraint refers to an unknown pointer because the difference of two pointers is considered an integer.

\subsection{Steensgaard Analysis}
Steensgaard Alias Analysis is another alias analysis technique that works across function calls without considering program flow \cite{Steensgaard}. Steensgaard analysis is based off an abstract pointer-based language that includes pointer operations, n-ary operators, dynamic memory, and functions. Type inference rules are used to generate alias sets for the pointer variables in the program. Each statement is initially processed once to generate the initial set of pointers; these values are stored in a union-find data structure, and are combined in alias sets in subsequent join operations. The resulting algorithm was found to be linear in space complexity and almost linear in terms of time complexity.

In the example program in Figure \ref{fig:codex}, shown above, the types for \texttt{a}, \texttt{b}, \texttt{c}, \texttt{x}, \texttt{y}, and \texttt{z} are discovered as distinct values in the first pass. After processing each statement once in the second pass of the algorithm, \texttt{x} refers to \texttt{a}, \texttt{y} refers to \texttt{a} and \texttt{b}, and \texttt{z} refers to \texttt{a} and \texttt{b}. Thus, all three pointer variables are found to alias under this analysis. Program flow is not captured well in this analysis, as possible aliases are always added for pointers instead of changed; in this example, \texttt{z} should only refer to \texttt{a} at the end of the program, and \texttt{y} only has an defined alias if the statically constant value for \texttt{c} is accounted for, which is usually not the case for an alias analysis. Additionally, for this analysis, the operands of a primitive operation, such as addition and subtraction, have the same type as the destination variable. This also results in inaccuracies, as shown in lines 10 and 13.

\subsection{LLVM Basic Alias Analysis}
The LLVM infrastructure features a basic alias analysis implementation available for use with compiler implementations \cite{llvmaa}. This alias analysis is local per function and depends on a series of heuristics to determine which pointers alias. For this analysis, distinct global variables, local variable declarations, and heap memory can never alias. Additionally, such values never alias the null pointer \cite{llvmaa}. Similarly, differing structure fields and array references that are statically different do not alias. Some C standard library functions are assumed to either never access program memory, or only access read-only memory. Pointers that refer to constant global values, such as strings, are said to point to constant memory. Finally, function calls cannot access local variables that never escape from the function that allocate them.

\subsection{Automatic Reference Counting}
Originally developed for the Objective-C programming language, Automatic Reference Counting (ARC) is a system of keeping track of allocated objects within a program \cite{ARC}. Dynamically allocated objects are given a reference count and a class based on its ownership, such as strong or weak ownership. To prevent memory leaks or accidental deallocations, objects are retained to add owners, and released to remove owners; objects with no owners are deallocated, and their pointers are set to null. Operations that refer to object pointers, such as reads, writes, initialization, destruction, and moving, are given different rules depending on the object's ownership type. While objects do not exist in C, ARC-based mechanisms can be applied to track pointer references, and are used in the LLVM infrastructure as part of an alias analysis that can be used for program optimizations.

\section{Evaluating Alias Analyses}
The first nearest analogue to this thesis's work can be found in Michael Hind and Anthony Pioli's research report, which attempts to measure several different alias analysis techniques under the same conditions and performance metrics \cite{Hind_old}. Specifically, Hind and Pioli explore three different techniques with varying degrees of precision and efficiency: Flow Insensitive Analysis, Flow Sensitive Analysis, and Flow Insensitive Analysis with Kill Information. These techniques are performed on input programs that are broken down into Control Flow Graphs (CFGs), and sets of pointer aliases are calculated at varying degrees of granularity. The Flow Insensitive analysis calculates possible aliases for variables across the entire function, with the Flow Insensitive Analysis with Kill including additional information about pointer definition and usage intended to improve the Flow Insensitive Analysis. On the other hand, the Flow Sensitive Analysis creates two alias sets for each CFG node, reflecting possible changes in the program due to control flow constructs. All three alias analyses are run on fourteen benchmark C programs, and precision is defined as the number of possible objects, or values, that a given pointer could refer to. Additional statistics are also collected from running these benchmarks, including the execution time of each analysis technique, distinctions between pointers used for reading or writing, and the type of pointer within the context of the program, such as local variables, global variables, formal parameters, and heap variables.

After running the benchmarks, the authors found that additional kill information did not improve the precision of Flow Insensitive analysis. The authors also found that the Flow Insensitive analysis was at least as precise, if not more so, than the Flow Sensitive analysis in half of the benchmarks used. The authors attribute this discrepancy to three possible causes: the first is that Flow Sensitive analysis becomes less precise as the size of a CFG increases, the second is that the consideration of formal and actual parameters is the same for Flow-Sensitive and Flow-Insenstive analyses, and the third is that pointers are often not modified in ways that would require the additional overhead needed for Flow-Sensitive analysis. The authors also propose efficiency improvements for alias analysis techniques, namely sharing alias sets between CFG nodes using Sparse Evaluation Graph (SEG) nodes to save space and reduce overhead in traversing the CFG, using sorted worklists to traverse CFG nodes, and only propagating alias relations that can be reached from a given function call. All of these improvements were shown to speed up the alias analyses in varying degrees. While this report does provide a detailed method of evaluating alias analysis techniques, its definition of precision is limited by the static nature of alias analysis techniques. Thus, there is no additional confirmation on whether a given alias is accurate with respect to the actual program.

Hind and Pioli produced another report evaluating various alias analysis techniques that expands on their previous work \cite{Hind}. This time, they examined six different context-insensitive analysis techniques. Four of these are flow-insensitive, one is flow-sensitive, and one is flow-insensitive but with additional kill information. The first technique, Address Taken (AT) analysis, is flow insensitive and computes a single global alias set for all objects in the program that were assigned to another variable. With its linear time complexity and limited precision, AT served as a baseline technique for comparison with the other techniques. The next technique, Steensgaard (ST) analysis, is a flow-insensitive analysis that computes a single union/find alias set in a single pass in almost linear time. The next flow-insensitive technique, Andersen (AN) analysis, implements Andersen's algorithm; normally, this algorithm uses constraint solving, but in the interest of efficiency, the analysis technique used in this report uses an iterative dataflow. Two flow-insensitive techniques proposed by Burke et al (B1 and B2) calculate local alias sets for each function call, with B2 including additional kill information for variable definition and usage. The final flow-sensitive algorithm proposed by Choi et al (CH) operates similarly to the B1 technique, but at the level of SEG nodes insteads of at the function level.

For this report, Hind and Pioli used twenty-four benchmark C programs, varying from under 1000 to almost 30,000 lines of code. These benchmarks themselves are measured through the resulting CFG's that are created for the pointer analyses, such as the number of CFG nodes, the number of function calls, and the number of heap allocations. As with the previous report, precision for each analysis technique is defined by the number of possible aliases for each given pointer. In addition to precision, execution time, memory usage, and the number of pointer reads and writes are measured for each analysis technique.

After running the benchmarks, both AT and ST were found to be efficient in terms of speed and memory usage. ST was significantly more precise than AT, especially as programs increased in size, with only minimal increases in overhead. AN and B1 varied in comparison to each other, with one significantly outperforming the other, and vice versa, in different benchmarks. The B2 analysis was consistently slower than B1, and the CH analysis was generally significantly slower, save for some benchmarks. The AN, B1, and B2 analyses had the same level of precision as one another, and were comparable with the CH analysis for many of the benchmarks. Thus, the additional kill information in B2 was again, not found to provide any significant benefit in increasing precision. As with the previous report, the benchmarks did not encounter the types of statements that would benefit from flow-sensitive analysis. Additionally, the CH analysis's memory usage was several times higher than that of its flow-insensitive counterparts, even after additional optimizations used to reduce its memory footprint. The speed of pointer analysis was found to be dependent on both program size and the number of propagated alias relations throughout a program's graph. Because this report is an expansion on a previous experiment, it also possesses the same limitations as the previous experiments, namely the definition of precision being limited by purely static analysis techniques.

\section{Proposed Analysis Techniques}

Because much of the overhead behind inclusion-based pointer analysis is due to the size of the generated constraint graph describing the relationship between pointer aliases, Hardekopf and Lin proposed two new algorithms to detect cycles within constraint graphs to reduce the size of the graph \cite{Hardekopf}. The first method, Lazy Cycle Detection (LCD), occurs when an alias set is propagated across nodes in the constraint graph; LCD checks the constraint graph for cycles based on two conditions; the first is whether or not two alias sets are identical, and the second is whether or not the graph edge related to the current pointer relation was searched previously. The second method, Hybrid Cycle Detection (HCD) performs a static analysis of the program before the actual pointer analysis to create a constraint graph and collapse any possible cycles; this preprocessing reduces the number of traverals performed by the actual pointer analysis, and provides additional information about which pointers might be part of a cycle, even if its alias sets are incomplete after performing HCD. LCD and HCD are evaluated with other comparative optimization algorithms for inclusion-based pointer analysis in five C benchmarks. In addition to the overall reduced number of constraints for each benchmark, the execution time and memory usage is measured for each algorithm. For the benchmarks, HCD is measured both by itself and in combination with the other algorithms. As individual algorithms, LCD and HCD had comparable execution times with the other algorithms. However, while LCD's memory usage was on par with the other algorithms, due to its preprocessing nature, HCD by itself could not complete all of the benchmarks due to running out of memory. When used in tandem, LCD and HCD significantly outperformed the other algorithms in terms of speed, with minimal decreases in memory usage. HCD also provided similar performance improvements when used in conjunction with the other algorithms.

One of the alias analysis techniques used in Hardekopf and Lin's experiments to compare against their proposed algorithms was a context-insensitive pointer analysis method developed by Pearce et al to account for fields and function pointers in an efficient, precise manner \cite{Pearce}. Previously, context-insensitive pointer analyses lacked the constraint types necessary to accurately reflect references to fields within user-defined structures; aggregate types were generally treated as a single variable, or treated as a distinct set of fields for either a unique instance of an aggregate or for all aggregates of the same type. Additionally, function pointers lacked any particular notation that could be used in an elegant or efficient manner. To account for these shortcomings, the authors introduced pointer constraints that included integer offsets, along with inference rules that utilize these constraints. These offsets can be used to model aggregate fields, and functions based on their addresses and parameters, and are treated as edge weights in a constraint graph. After running both field-sensitive and field-insensitive versions of their new analysis technique on seven benchmarks, Pearce et al found that field-sensitive analysis offered more precision, but at the cost of increased execution time. However, the increase in precision was also found to decrease with larger programs.
