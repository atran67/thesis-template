\chapter{Future Work}

While this thesis is an exploratory study into the effectiveness of alias analyses, there are some limitations that, given more time, could be addressed in subsequent studies. Most of the future work related to this thesis involves elaborating upon various aspects of the study to be more specific at gathering and quantifying data, along with exploring additional alias analyses, programs, and interesting program statistics.

\section{Additional Alias Analyses}
Additional studies could be conducted to measure the effectiveness of other alias analyses. The LLVM Optimizer features several other built-in alias analyses that were not included in this thesis that could be examined. Given more time, more recently proposed alias analyses could be implemented to work at the LLVM level for similar measurement.

\section{Improved Instrumentation}
The instrumentation for this thesis consists of printing out data for memory access instructions. Because of the I/O overhead associated with printing out information, this is not the most efficient method of instrumentation; when considering the large number of memory access instructions within the unoptimized benchmark programs, the instrumented programs are several orders of magnitude slower than their uninstrumented counterparts. More time could be dedicated to developing or modifying a framework for logging a program's memory accesses at the LLVM level, similar to how debuggers handle programs in isolated environments.
