As software projects become larger and more complex, software optimization at that scale is only feasible through automated means. One such component of software optimization is alias analysis, which attempts to determine which variables in a program refer to the same area in memory; this is used to relocate instructions to improve performance without interfering with program execution. Several alias analyses have been proposed over the past few decades, with varying degrees of precision and time and space complexity, but few studies have been conducted to compare these techniques with one another, nor to measure with program data to confirm their accuracy. Normally, this is out of the scope of alias analyses because these processes are static, and can only rely upon the input source code. We address these limitations by instrumenting several benchmarks and combining their data with commonly used alias analyses to objectively measure the accuracy of those analyses. Additionally, we also gather additional program statistics to further determine which programs are the most suitable for evaluating subsequent alias analysis techniques.

